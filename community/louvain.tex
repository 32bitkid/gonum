% Copyright ©2015 The gonum Authors. All rights reserved.
% Use of this source code is governed by a BSD-style
% license that can be found in the LICENSE file.

\documentclass{article}

\usepackage{amsmath,amsfonts}
\usepackage[margin=4cm]{geometry}

\begin{document}

The algorithm attempts to find communities (highly connected sub-graphs),
and it does this by minimising the modularity function
\begin{equation}
	Q = \frac{1}{2m}\sum_i\sum_j\left[ A_{ij} - \gamma \frac{k_ik_j}{2m} \right] \delta(c_i,c_j),
\end{equation}
where $A_{ij}$ is the edge weight between nodes $i$ and $j$, 
$c_i$ is the current community to which node $i$ belongs,
$\gamma$ is a tuning parameter,
\begin{equation}
m = \frac{1}{2}\sum_i\sum_jA_{ij},
\end{equation}
\begin{equation}
k_i = \sum_j{A_{ij}},
\end{equation}
and 
\begin{equation}
	\delta(c_i,c_j) = \left \{ \begin{array}{ll}
	1 & \text{if} \quad  c_i = c_j \\
	0 & \text{otherwise} \end{array} \right ..
\end{equation}

The algorithm finds a hierarchical community structure by iterating 
between two phases:
\begin{enumerate}
	\item It finds a set of communities that minimise $Q$.
	\item It constructs a new graph, whose nodes are the communities 
	found in the preceding phase one step.
\end{enumerate}
Each iteration of these two phases is called a `pass'.
In this way, the algorithm obtains a nested community structure, 
where at each level $Q$ is minimised for the relevant graph.
We consider this process in more detail, in particular looking 
at phase one first in the first pass, when each node is a single
node, and then how this generalises to later passes when each node 
is a community.


\section{Initial Pass}
\label{sec:initialPass}

The initial pass is simple as the initial pass uses the original graph, 
and in all following passes graphs constructed in the previous pass's 
phase two are used.
Here we will consider this initial simple formulation for phase one, and
in Section~\ref{sec:laterPasses} we consider how this generalises for 
passes two and onwards. 
Phase one works by initially allocating each node to a separate community,
and then iterating through each node $a$ and checking if moving it into 
a different community $\beta$ will reduce $Q$.
If there are possible moves that will reduce $Q$, $a$ is moved into the 
the community which will generate the largest reduction in $Q$. 
This process is continued until there are no moves left to reduce $Q$ 
further, meaning a local minima for $Q$ has been achieved. 
Then the algorithm moves to phase two (constructing a new graph where
each node in the new graph is a community in the old graph). 

Note that we assume the original graph to be simple and undirected.
First, we introduce some notation that will be useful:
Let $c_i$ denote the community to which node $i$ currently belongs, 
and let $\alpha$ be the community that the node $a$ mentioned above
belongs to ($c_a = \alpha$), i.e.
\begin{equation}
	\alpha = \{i | c_i = c_a\}.
\end{equation}
Then we define
\newcommand{\Stot}[1]{\Sigma_{\text{tot}}^{#1}}
\begin{equation}
	\Stot{\alpha} = \sum_{i \in \alpha}\sum_{j}A_{ij} = \sum_{i \in \alpha}k_i,
\end{equation}
\newcommand{\kin}[2]{k_{#1}^{#2}}
\begin{equation}
	\kin{i}{\alpha} = \sum_{j \in \alpha}A_{ij},
\end{equation}
and
\newcommand{\Sin}[1]{\Sigma_{\text{in}}^{#1}}
\begin{equation}
	\Sin{\alpha} = \sum_{i \in \alpha}\sum_{j \in \alpha}A_{ij} = \sum_{i \in \alpha}\kin{i}{\alpha}.
\end{equation}

We are interested in how $Q$ will change if we move a node $a$ from its 
current community $\alpha$, to a new community $\beta$.
This will have two effects, it will remove the terms from $Q$ 
related to $a$ in $\alpha$, which we will call $Q^-$ and it will add terms 
in $\beta$ related to $a$, which we will call $Q^+$. 
The total change in $Q$ caused by the movement of $a$ from $\alpha$ to $\beta$ is 
\begin{equation}
	\Delta Q = Q^{+} - Q^{-},
\end{equation}
where
\begin{align*}
Q^- &= \frac{1}{2m}\left[ \left( A_{aa} - \gamma \frac{k_a^2}{2m} \right) 
+ 2\sum_{i \in \alpha, \, i \neq a} \left( A_{ia} - \gamma \frac{k_ik_a}{2m} \right) \right] \\
	&= \frac{1}{2m}\left[ \left( A_{aa} - \gamma \frac{k_a^2}{2m} \right) 
+ 2 \left( \kin{a}{\alpha} -A_{aa}\right) - \gamma \frac{2k_a}{2m}\sum_{i \in \alpha, \, i \neq a} k_i \right] \\
	&= \frac{1}{2m}\left[ \left( A_{aa} - \gamma \frac{k_a^2}{2m} \right) 
+ 2 \left( \kin{a}{\alpha} -A_{aa}\right) - \gamma \frac{2k_a}{2m}\left( \Stot{\alpha} - k_a \right)  \right], \\
\end{align*}
and
\begin{align*}
Q^+ &= \frac{1}{2m}\left[ \left( A_{aa} - \gamma \frac{k_a^2}{2m} \right) 
+ 2\sum_{i \in \beta} \left( A_{ia} - \gamma \frac{k_ik_a}{2m} \right) \right] \\
		&= \frac{1}{2m}\left[ \left( A_{aa} - \gamma \frac{k_a^2}{2m} \right) 
+ 2\kin{a}{\beta} - \gamma \frac{2k_a}{2m}\sum_{i \in \beta} k_i \right] \\
		&= \frac{1}{2m}\left[ \left( A_{aa} - \gamma \frac{k_a^2}{2m} \right) 
+ 2\kin{a}{\beta} - \gamma \frac{2k_a\Stot{\beta}}{2m} \right]. \\
\end{align*}
The first term in both these expressions ($Q^-$ and $Q^+$) is the same, and so cancels:
\begin{equation}
\Delta Q = \frac{1}{2m}\left[ \left( 2\kin{a}{\beta} - \gamma \frac{2k_a\Stot{\beta}}{2m} \right) 
		- \left( 2 \left( \kin{a}{\alpha} -A_{aa}\right) - \gamma \frac{2k_a}{2m}\left( \Stot{\alpha} - k_a \right) \right) \right]. 
\end{equation}
 
\section{Later Passes}
\label{sec:laterPasses}

In phase two a `meta-graph' is constructed where nodes correspond to 
the communities found in the preceding phase one step, and edge weight
between two such communities (nodes, in the meta-graph)
$\alpha$ and $\beta$ are defined to be
\begin{equation}
	A_{\alpha \beta}^* = \sum_{i \in \alpha}\sum_{j \in \beta}A_{ij}.
	\label{eqn:Aij*}
\end{equation} 
Note that $i$ and $j$ refer to nodes in the original graph, not nodes 
in the previous graph, and so holds any meta-graph, not just the first.
Also note that this definition of $A^*_{\alpha \beta}$ allows for 
$A^*_{\alpha \alpha}$ to be non-zero, infact
\begin{equation}
A_{\alpha \alpha}^* = \sum_{i \in \alpha}\sum_{j \in \alpha}A_{ij} = \Sin{\alpha}.
\end{equation} 

In this newly constructed graph, $\alpha$ and $\beta$ are nodes, but 
also refer to communities (sets of nodes) in the original graph, and I 
use these two interpretations interchangeably.
This should be the only ambiguous bit of notation in this document, I hope.

The results of Section~\ref{sec:initialPass} generalise to these meta-graphs,
and the generalised results mirror those of Section~\ref{sec:initialPass} closely
-- I distinguish the new results from those of Section~\ref{sec:initialPass} by a 
superscript $*$.
$i$ and $j$ to denote nodes of the original graph as in Section~\ref{sec:initialPass}, 
and use $z$ and $w$ to denote nodes of the meta-graph (communities of the original).
I use the same notation as Section~\ref{sec:initialPass}, $c_z$, to 
denote the community to which node $z$ of the meta-graph currently belongs, 
and let $\mathfrak{a}$ be the community that the node $\alpha$ belongs to 
($c_{\alpha} = \mathfrak{a}$), i.e.
\begin{equation}
	\mathfrak{a} = \{z | c_z^* = c_{\alpha}^* \}.
\end{equation}

Given this notation, we can observe that
\begin{equation}
m^* = \frac{1}{2}\sum_{z}\sum_{w}{A_{zw}^*} = \frac{1}{2}\sum_{z}\sum_{w}{\sum_{i \in z}\sum_{j \in w}A_{ij}} = \frac{1}{2}\sum_i\sum_jA_{ij} = m,
\end{equation}
\begin{equation}
k_{z}^* = \sum_{w}{A_{zw}^*} =  \sum_{w}{\sum_{i \in z}\sum_{j \in w}A_{ij}} = \sum_{i \in z}\sum_{j}A_{ij} = \Stot{z},
\end{equation}
\begin{equation}
	\Stot{\mathfrak{a} *} = \sum_{z \in \mathfrak{a}}\sum_{w}A_{zw}^* = \sum_{z \in \mathfrak{a}}k_z^* = \sum_{z \in \mathfrak{a}}\Stot{z},
\end{equation}
\begin{equation}
	\kin{z}{\mathfrak{a} *} = \sum_{w \in \mathfrak{a}}{A_{zw}^*} = \sum_{w \in \mathfrak{a}}{\sum_{i \in z}\sum_{j \in w}A_{ij}},
\end{equation}
and
\begin{equation}
\Sin{\mathfrak{a} *} = \sum_{z \in \mathfrak{a}}\sum_{w \in \mathfrak{a}}A_{zw}^* = \sum_{z \in \mathfrak{a}}\kin{z}{\mathfrak{a} *} = \sum_{z \in \mathfrak{a}}\sum_{w \in \mathfrak{a}}{\sum_{i \in z}\sum_{j \in w}A_{ij}}.
	%\label{eqn:Sin}
\end{equation}

If we let $\mathfrak{b}$ denote the community to which we are considering moving $\alpha$,
then the expression for $\Delta Q$ from Section~\ref{sec:initialPass} trivially generalises to
\begin{equation}
\Delta Q = \frac{1}{2m}\left[ \left( 2 \kin{\alpha}{\mathfrak{b} *} - \gamma \frac{2k_{\alpha}^*\Stot{\mathfrak{b} *}}{2m} \right) 
		- \left( 2\left( \kin{\alpha}{\mathfrak{a} *} - A_{\alpha \alpha}^* \right) - \gamma \frac{2k_{\alpha}^*}{2m} \left( \Stot{\mathfrak{a} *} - k_{\alpha}^* \right ) \right) \right] \\
\end{equation}

\section{Directed Graphs}
\label{sec:directedGraphs}

It is of interest to consider how this generalises to directed graphs.
If we are to treat incoming and outgoing nodes equally, there are two 
thoughts on how to extend the algorithm to directed graphs: to generalise
the expressions to allow for directed graphs, or to construct an undirected
graph by averaging the weights of edges in opposite directions.

The suggested extension to directed graphs is to modify the expressions 
involved by adding the `from' case and the `to' case for each term. 
If we let $B_{ij}$ be the edge weight between nodes $i$ and $j$ in 
the directed graph of interest, these extended expressions look like
\begin{equation}
m = \frac{1}{2}\left ( \sum_i\sum_jB_{ij} + \sum_i\sum_jB_{ji}\right)  = \frac{1}{2}\sum_i\sum_j \left( B_{ij} + B_{ji} \right) ,
\end{equation}
\begin{equation}
k_i = \sum_jB_{ij} + \sum_jB_{ji} = \sum_j{\left( B_{ij} + B_{ji} \right)},
\end{equation}
and similarly
\begin{equation}
	Q = \frac{1}{2m}\sum_i\sum_j\left[ \left( B_{ij} + B_{ji} \right) - \gamma \frac{k_ik_j}{2m} \right] \delta(c_i,c_j).
\end{equation}

Note how this is equivalent to constructing an undirected graph with
edge weights
\begin{equation}
A_{ij} = B_{ij} + B_{ji}.
\end{equation}
However the other suggestion is to averaging the directed edges to make
an undirected graph, i.e. to use
\begin{equation}
A_{ij} = \frac{B_{ij} + B_{ji}}{2}.
\end{equation}
This raises an important question: does scaling all edge weights across 
the entire graph by a constant affect the results of the algorithm?
Hopefully not, but worth checking.
We can follow this through the results above by applying the 
substitution $A_{ij}' = pA_{ij}$ ($p \in \mathbb{R}$) and using ${}'$s
to denote the transformed expressions, we get:
\begin{equation}
m' = \frac{1}{2}\sum_i\sum_jpA_{ij}  = p\frac{1}{2}\sum_i\sum_j A_{ij} = pm ,
\end{equation}
\begin{equation}
k_i' = \sum_j{pA_{ij}} = p\sum_j{A_{ij}} = pk_i,
\end{equation}
and so
\begin{align*}
	Q' &= \frac{1}{2pm}\sum_i\sum_j\left[ pA_{ij} - \gamma \frac{pk_ipk_j}{2pm} \right] \delta(c_i,c_j) \\
	&= \frac{1}{2m}\sum_i\sum_j\left[ A_{ij} - \gamma \frac{k_ik_j}{2m} \right] \delta(c_i,c_j) \\
	&= Q
\end{align*}
Note that as we have shown $Q' = Q$ there is no need to go into the remainder of the terms 
involved in the algorithm, as they all derive from $Q$.


\end{document}
